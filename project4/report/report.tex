\documentclass[12pt]{article}

\usepackage[utf8]{inputenc}
\usepackage{amsmath}
\usepackage{amssymb}
\usepackage{graphicx}
\usepackage{hyperref}
\usepackage{geometry}
\usepackage{algorithm}
\usepackage{algpseudocode}
\usepackage{algorithmicx}

\geometry{a4paper, margin=1in}

\title{Project 4}
\author{Brock Gregersen}
\date{\today}

\begin{document}

\maketitle

\tableofcontents
\newpage

\section{Unrestricted Algrorithm}
\subsection{Data Structure}
My implementation of the unrestricted Needleman-Wunsch algorithm uses
a datastructure using an underlying $n \times m$ list which is created in $O(nm)$
time. It also saves the input as strings. This datastructure includes
constant time and space functions for setting and getting values, as well
as checking whether the input strings match at given indices. A function to
initialize the values on the top and left edges of the matrix is also included:

\begin{algorithm}[H]
    \caption{Matrix.\textsc{init\_values}}
    \begin{algorithmic}[1]
        \State \textbf{Input:} indel\_penalty
        \State matrix[0][0] $\gets$ 0
        \For{$i = 1$ to $n$} \Comment{runs $n$ times}
            \State matrix[i][0] $\gets$ i * indel\_penalty
        \EndFor
        \For{$j = 1$ to $m$} \Comment{runs $m$ times}
            \State matrix[0][j] $\gets$ j * indel\_penalty
        \EndFor
    \end{algorithmic}
\end{algorithm}

\textsc{init\_values} uses constant time operations inside loops that
run $n$ and $m$ times, respectively, leading to a time complexity of
$O(n + m)$. No additional space is used.

\subsection{Minimization}

My implementation of the unrestricted Needleman-Wunsch algorithm uses the
following algorithm to fill in the matrix and find the minimum score:

\begin{algorithm}[H]
    \caption{\textsc{minimize\_alignment}}
    \begin{algorithmic}[1]
        \State \textbf{Input:} str1, str2, indel\_penalty, match\_score, mismatch\_penalty
        \State matrix $\gets$ new Matrix(str1.length(), str2.length()) \Comment{$O(nm)$}
        \State matrix.\textsc{init\_values}(indel\_penalty) \Comment{$O(n + m)$}
        \For{$i = 1$ to $n$} \Comment{runs $n$ times}
            \For{$j = 1$ to $m$} \Comment{runs $m$ times}
                \State insert\_score $\gets$ matrix[$i-1$][$j$] + indel\_penalty
                \State delete\_score $\gets$ matrix[$i$][$j-1$] + indel\_penalty
                \If{\textsc{match}($i$, $j$)}
                    \State match\_score $\gets$ matrix[$i-1$][$j-1$] + match\_score
                \Else
                    \State match\_score $\gets$ matrix[$i-1$][$j-1$] + mismatch\_penalty
                \EndIf
                \State matrix[$i$][$j$] $\gets$ min(insert\_score, delete\_score, match\_score)
            \EndFor
        \EndFor
        \State \textbf{return:} matrix$[n][m]$
    \end{algorithmic}
\end{algorithm}

The time complexity only scales with the number of times the loops run,
leading to a time complexity of $O(nm)$. The only non constant space used
is the matrix and the input strings, which are $O(nm)$ and $O(n + m)$,
respectively, giving $O(nm)$ space complexity.


\subsection{Backtracking}

The algorithm then backtracks through the matrix to find the optimal
alignment using the following algorithm:

\begin{algorithm}[H]
    \caption{\textsc{backtrack}}
    \label{alg:backtrack}
    \begin{algorithmic}[1]
        \State alignment1 $\gets$ \texttt{""}
        \State alignment2 $\gets$ \texttt{""}
        \State $i \gets$ length(str1)
        \State $j \gets$ length(str2)
        \While{$i > 0$ \textbf{or} $j > 0$} \Comment{runs at most $n + m$ times}
            \If{matrix$[i][j]$ = matrix$[i-1][j-1] + \text{indel\_penalty}$ \textbf{and} $\textsc{match}(i, j)$}
                \State alignment1 $\gets$ str1[$i$] + alignment1
                \State alignment2 $\gets$ str2[$j$] + alignment2
                \State $i \gets i - 1$
                \State $j \gets j - 1$
            \ElsIf{matrix$[i][j]$ = matrix$[i-1][j-1] + \text{indel\_penalty}$}
                \State alignment1 $\gets$ str1[$i$] + alignment1
                \State alignment2 $\gets$ str2[$j$] + alignment2
                \State $i \gets i - 1$
                \State $j \gets j - 1$
            \ElsIf{matrix$[i][j]$ = matrix$[i][j-1] + \text{indel\_penalty}$}
                \State alignment1 $\gets$ \texttt{"-"} + alignment1
                \State alignment2 $\gets$ str2[$j$] + alignment2
                \State $j \gets j - 1$
            \ElsIf{matrix$[i][j]$ = matrix$[i-1][j] + \text{indel\_penalty}$}
                \State alignment1 $\gets$ str1[$i$] + alignment1
                \State alignment2 $\gets$ \texttt{"-"} + alignment2
                \State $i \gets i - 1$
            \EndIf
        \EndWhile
        \State \textbf{return:} alignment1, alignment2
    \end{algorithmic}
\end{algorithm}

All operations inside the loops are constant time, and the loops run at most
$n + m$ times, leading to a time complexity of $O(n + m)$. The only additional
space used is for the two strings, which each scale with the size of $n$ and $m$,
leading to a space complexity of $O(n + m)$. In addition to the time and space
complexity of finding the minimum score, both the time and space complexity of
the unrestricted algorithm is $O(nm)$.

\newpage

\section{Banded Algorithm}

\subsection{Data Structure}

Similar to the unrestricted algorithm, my implementation of the banded
Needleman-Wunsch algorithm uses a 2D list as the underlying data structure,
however in this case, the matrix is of size $n \times k$ where $k$ is the
bandwith, or $2d + 1$. As with the unrestricted algorithm, constant time
and space functions are implemented for setting and getting values, as well as
checking whether the input strings match at given indices. This datastructure
also abstracts the $n \times k$ matrix so that values can be accessed using
the same indices as the unrestricted algorithm. A function to
initialize the values on the top and left edges of the matrix is also included:
\begin{algorithm}[H]
    \caption{Matrix.\textsc{init\_values}}
    \begin{algorithmic}[1]
        \State \textbf{Input:} indel\_penalty
        \State matrix[0][0] $\gets$ 0
        \For{$i = 1$ to $d$} \Comment{runs $d$ times}
            \State matrix[i][0] $\gets$ $i$ * indel\_penalty
        \EndFor
        \For{$j = 1$ to $d$} \Comment{runs $d$ times}
            \State matrix[0][j] $\gets$ $j$ * indel\_penalty
        \EndFor
    \end{algorithmic}
\end{algorithm}

\textsc{init\_values} uses constant time operations inside loops that
run $d$ times, leading to a time complexity of $O(d)$. No additional space
is used.

\subsection{Minimization}
The following algorithm is used to fill in the matrix and find
the minimum score:

\begin{algorithm}[H]
    \caption{\textsc{minimize\_alignment}}
    \begin{algorithmic}[1]
        \State \textbf{Input:} str1, str2, indel\_penalty, match\_score, mismatch\_penalty
        \State matrix $\gets$ new Matrix(str1.length(), str2.length()) \Comment{$O(nd)$}
        \State matrix.\textsc{init\_values}(indel\_penalty) \Comment{$O(d)$}
        \For{$i = 1$ to $n$} \Comment{runs $n$ times}
            \For{$j = i - d$ to $i + d$} \Comment{runs $2d + 1 = k$ times}
                \State insert\_score $\gets$ matrix[$i-1$][$j$] + indel\_penalty
                \State delete\_score $\gets$ matrix[$i$][$j-1$] + indel\_penalty
                \If{\textsc{match}($i$, $j$)}
                    \State match\_score $\gets$ matrix[$i-1$][$j-1$] + match\_score
                \Else
                    \State match\_score $\gets$ matrix[$i-1$][$j-1$] + mismatch\_penalty
                \EndIf
                \State matrix[$i$][$j$] $\gets$ min(insert\_score, delete\_score, match\_score)
            \EndFor
        \EndFor
        \State \textbf{return:} matrix$[n][m]$
    \end{algorithmic}
\end{algorithm}

The time complexity only scales with the number of times the loops run,
leading to a time complexity of $O(n(d + 1)) =  O(nd)$. The only non
constant space used is the matrix and the input strings, which are $O(nd)$
and $O(n + m)$, respectively, giving $O(nd)$ space complexity, assuming
$n = m$.

\subsection{Backtracking}
The algorithm then backtracks through the matrix to find the optimal
alignment using an algorithm identical to the backtracking algorithm used
in the unrestricted algorithm, shown in 
\textbf{\hyperref[alg:backtrack]{Algorithm~\ref*{alg:backtrack}}}.
The $O(n + m)$ time and space complexity of the backtracking algorithm in 
addition to the $O(nd)$ time and space complexity of the banded algorithm
gives a total time complexity of $O(nd)$ and a total space complexity of $O(nd)$
for the banded algorithm, assuming $n = m$.



\end{document}