\documentclass[12pt]{article}

% Packages
\usepackage{amsmath}
\usepackage{amssymb}
\usepackage{hyperref}
\usepackage{booktabs}
\usepackage{threeparttable}
\usepackage[top=1in, bottom=1in, left=1in, right=1in]{geometry} % Adjust margins

% Packages
\usepackage{amsmath}
\usepackage{amssymb}
\usepackage{hyperref}
\usepackage{booktabs}
\usepackage{threeparttable}

% Title
\title{CS312 Project 1 Report}
\author{Brock Gregersen}
\date{\today}

\begin{document}

\maketitle

\section{Algorithm Analysis}

\begin{table}[h!]
\centering
\begin{threeparttable}
\begin{tabular}{@{}lcc@{}}
\toprule
\textbf{Algorithm}                  & \textbf{Time Complexity}   & \textbf{Space Complexity} \\ \midrule
\texttt{mod\_exp()}                 & $O(n^{3})$                 & $O(n)$              \\
\texttt{fermat()}                   & $O(kn^{3})$                & $O(kn)$                    \\
\texttt{miller\_rabin()}            & $O(kn^{4})$                & $O(kn)$                     \\
\texttt{extended\_euclid()}         & $O(n^{3})$                 & $O(n)$                      \\
\texttt{generate\_random\_prime()}\tnote{*}   
                                    & $O(n^{5})$                 & $O(n)$                      \\
\texttt{rsa\_generate\_keypair()}\tnote{*} 
                                    & $O(n^{5})$                 & $O(n)$                      \\ \bottomrule
\end{tabular}
\begin{tablenotes}
\item[*] Using Miller-Rabin primality test.
\end{tablenotes}
\end{threeparttable}
\caption{Time and Space Complexity of Algorithms for $n$ bits and $k$ iterations}
\label{tab:complexities}
\end{table}

\section{Correctness Probability of Primality Tests}

\subsection{Fermat Primality Test}
For Fermat's primality test, at most $\frac{n}{2}$ integers less than $n$ give a false positive, so given $k$ random integers, assuming that $k$ is significantly smaller than $n$, the probability that all are false positives is at most $\frac{1}{k^{2}}$.

\subsection{Miller-Rabin Primality Test}
For Miller-Rabin's primality test, at most $\frac{n}{4}$ integers less than $n$ give a false positive, so given $k$ random integers, assuming that $k$ is significantly smaller than $n$, the probability that all are false positives is at most $\frac{1}{k^{4}}$. 

\end{document}
% Title
\title{CS312 Project 1 Report}
\author{Brock Gregersen}
\date{\today}

\begin{document}

\maketitle

\section{Algorithm Analysis}

\begin{table}[h!]
\centering
\begin{threeparttable}
\begin{tabular}{@{}lcc@{}}
\toprule
\textbf{Algorithm}                  & \textbf{Time Complexity}   & \textbf{Space Complexity} \\ \midrule
\texttt{mod\_exp()}                 & $O(n^{3})$                 & $O(n)$              \\
\texttt{fermat()}                   & $O(kn^{3})$                & $O(kn)$                    \\
\texttt{miller\_rabin()}            & $O(kn^{4})$                & $O(kn)$                     \\
\texttt{extended\_euclid()}         & $O(n^{3})$                 & $O(n)$                      \\
\texttt{generate\_random\_prime()}\tnote{*}   
                                    & $O(n^{5})$                 & $O(n)$                      \\
\texttt{rsa\_generate\_keypair()}\tnote{*} 
                                    & $O(n^{5})$                 & $O(n)$                      \\ \bottomrule
\end{tabular}
\begin{tablenotes}
\item[*] Using Miller-Rabin primality test.
\end{tablenotes}
\end{threeparttable}
\caption{Time and Space Complexity of Algorithms for $n$ bits and $k$ iterations}
\label{tab:complexities}
\end{table}

\section{Correctness Probability of Primality Tests}

\subsection{Fermat Primality Test}
For Fermat's primality test, at most $\frac{n}{2}$ integers less than $n$ give a false positive, so given $k$ random integers, assuming that $k$ is significantly smaller than $n$, the probability that all are false positives is at most $\frac{1}{k^{2}}$.

\subsection{Miller-Rabin Primality Test}
For Miller-Rabin's primality test, at most $\frac{n}{4}$ integers less than $n$ give a false positive, so given $k$ random integers, assuming that $k$ is significantly smaller than $n$, the probability that all are false positives is at most $\frac{1}{k^{4}}$. 

\end{document}